% !TeX TXS-program:compile = txs:///pdflatex/
% !TeX TXS-program:bibliography = txs:///biber
% !TeX program = pdflatex
% !BIB program = biber




%%%%%%%%%%%%%%%%%%%%%%%%%%%%%%%%%%%%%%%%%%%%%%%%
%%  CITATION COMMANDS AND BIBLIOGRAPHY STYLE  %%
%%%%%%%%%%%%%%%%%%%%%%%%%%%%%%%%%%%%%%%%%%%%%%%%


% AER/JEL/JEP style

\usepackage[backend=biber, natbib=true, bibencoding=inputenc, bibstyle=authoryear, citestyle=authoryear-comp, mincitenames=1, maxcitenames=3, minbibnames=99, maxbibnames=99, uniquename=false, uniquelist=true, backref=true, backrefstyle=three, doi=true, isbn=false, dashed=false, sorting=ynt, sortcites=true, mergedate=true, dateabbrev=false, abbreviate=false, citetracker=true]{biblatex}
% sortcites sorts the in-text citations by year of publication
\DeclareBibliographyAlias{newspaper}{article}

% Full author list on first citation, ``et al.'' only from second citation onwards:
% https://tex.stackexchange.com/questions/48846/biblatex-et-al-beginning-from-second-citation
% ==>
%\AtEveryCitekey{\ifciteseen{}{\clearfield{namehash}}}  % Uncomment this ``so that an entry won't be part of a compact citation list the first time it is cited.''
\xpatchbibmacro{cite}
	{\printnames{labelname}}
	{\ifciteseen
		{\printnames{labelname}}
		{\printnames[][1-5]{labelname}}%
	}
	{}
	{}
\xpatchbibmacro{textcite}
	{\printnames{labelname}}
	{\ifciteseen
		{\printnames{labelname}}
		{\printnames[][1-5]{labelname}}%
	}
	{}
	{}
% <==

\let\citeorig\cite
\renewcommand{\cite}{\citet}
\renewcommand{\citealp}{\citeorig}

\defbibheading{subbibliography}[\refname]{%
	\section*{#1}%
	\sectionmark{#1}%
	\addcontentsline{toc}{section}{#1}%
}

\renewcommand{\bibfont}{\sffamily\small}
	% Reduce font size for the bibliography, make the font sans-serif.
\setlength{\bibindent}{\parindent}
\setlength{\bibitemsep}{0pt}

\DefineBibliographyStrings{english}{%
	andothers = {et~al\adddot},
	volume = {vol\adddot}
}

% Remove parentheses from the date:
\xpatchbibmacro{date+extradate}{%
	\printtext[parens]%
}{%
	\setunit*{\mkbibbold{\addperiod\space}}%
	\mdseries\selectfont\printtext%
}{}{}
% Make the author/editor name(s) bold ==>
% See https://tex.stackexchange.com/questions/41468/biblatex-textcite-author-formatting-vs-bibliography-author-formatting
\xpatchbibmacro{author}{\printnames{author}}{\mkbibbold{\printnames{author}\addperiod}}{}{}
%\xpretobibmacro{author}{\mkbibbold\bgroup}{}{}
%\xapptobibmacro{author}{\egroup}{}{}
\xpretobibmacro{editor+others}{\mkbibbold\bgroup}{}{}%
\xapptobibmacro{editor+others}{\egroup\clearname{editor}}{}{}%
\xpretobibmacro{translator+others}{\mkbibbold\bgroup}{}{}%
\xapptobibmacro{translator+others}{\egroup\clearname{translator}}{}{}
% Set up page range compression (e.g., 1034-1067 => 1034-67)
\setcounter{mincomprange}{100}
\setcounter{maxcomprange}{100000}
\setcounter{mincompwidth}{10}
\DeclareFieldFormat{pages}{\mkcomprange{#1}}
% Removes ``pp.'' from pages and compresses page ranges:
\DeclareFieldFormat[article, inbook, incollection, inproceedings, patent, thesis, unpublished, newspaper]{pages}{{\nopp\mkcomprange{#1}}} 
% Include comma also after first name in the case of only two authors:
\AtEveryBibitem{
	\renewcommand*{\finalnamedelim}{\finalandcomma\addspace\bibstring{and}\space}
	\renewcommand*{\finallistdelim}{\finalandcomma\addspace\bibstring{and}\space}
}
\xpretobibmacro{byeditor+others}  % Undo this for the listing of editors
	{%
		\renewcommand*{\finalnamedelim}{%
			\ifnumgreater{\value{liststop}}{2}{\finalandcomma}{}%
			\addspace\bibstring{and}\space%
		}%
	}
	{}
	{\DoNotContinue}

% \renewcommand*{\newunitpunct}{\addcomma\space}
\renewcommand*{\bibnamedash}{---{\kern -2.25pt}---{\kern -2.25pt}---\addcomma\space}

%\DeclareCiteCommand{\citejournal}
%  {\usebibmacro{prenote}}
%  {\usebibmacro{citeindex}%
%   \usebibmacro{journal}}
%  {\multicitedelim}
%  {\usebibmacro{postnote}}

% Remove ``in'' for journal and newspaper articles:
\renewcommand*{\intitlepunct}{\space}
\renewbibmacro{in:}{%
	\ifentrytype{article}%
		{}%
		{\ifentrytype{newspaper}%
			{}%
			{\printtext{\bibstring{in}\intitlepunct}}%
		}%
}

% How to handle DOIs and URLs
% ==>
\renewbibmacro*{doi+eprint+url}
{%
	\iffieldundef{doi}
	{}
	{\printfield{doi}}
	\newunit\newblock
	\iftoggle{bbx:eprint}
	{\usebibmacro{eprint}}
	{}%
	\newunit\newblock
	\iffieldundef{doi}  % If no DOI provided, use URL
	{\usebibmacro{url+urldate}}
	{}
}
\DeclareFieldFormat{url}{%
	% \setlength{\Urlmuskip}{0mu}%
	\Urlmuskip = 0mu\relax%
	\caps{URL}\addcolon\space\url{#1}%
}
% Link DOIs automatically to https://dx.doi.org/DOI:
\DeclareFieldFormat{doi}{%
	% \setlength{\Urlmuskip}{0pt}%
	\Urlmuskip = 0mu\relax
	\caps{DOI}\addcolon\space\href{https://dx.doi.org/#1}{\nolinkurl{#1}}%
}
% <==

% Make journal title italic:
\DeclareFieldFormat[article]{journaltitle}{%
	\mkbibemph{#1}%
	\iffieldundef{volume}{\addcomma}{}%
}

% Handling of newspaper articles
\DeclareFieldFormat[newspaper]{journaltitle}{%
	\mkbibemph{#1} \mkbibparens{\thefield{edition}}\addcomma\addspace%
	\iflanguage{ngerman}{%
		\addnbspace\thefield{day}\addperiod\addnbspace\mkbibmonth{\thefield{month}}\addspace\thefield{year}%
	}{% else: default to the USenglish version
		\mkbibmonth{\thefield{month}}\addnbspace\thefield{day}\addcomma\addspace\thefield{year}%
	}%
	\iffieldundef{volume}{\addcolon}{\addcomma}%
}
\DeclareFieldFormat[newspaper]{date}{%
	\thefield{year}%
}

% Format the ``month'' field:
\DeclareFieldFormat{month}{\mkbibmonth{#1}}

% Put the title of articles in quotation marks:
\DeclareFieldFormat[thesis, unpublished, report, misc, newspaper]{title}{\mkbibquote{#1}}
%% Make title of books italic:
%\DeclareFieldFormat[book]{title}{\mkbibemph{#1}\nopunct}

% Makes first character of document type uppercase:
\DeclareFieldFormat[thesis, unpublished, report, misc]{type}{\MakeCapital{#1}}

%% Remove punctuation after ``series'' field:
%\DeclareFieldFormat[incollection]{series}{#1\nopunct}

% Add ``vol.'' for books etc.:
\DeclareFieldFormat[book, inbook, incollection, inproceedings]{volume}{\bibstring{volume}\addnbspace#1\addcomma}
% Formatting VV\,(NN):
\DeclareFieldFormat[article]{volume}{%
	\iffieldundef{volume}{}{%
		\addspace#1%
		\iffieldundef{number}{\addcolon}{\nopunct}%
	}
}
\DeclareFieldFormat[article]{number}{\addnbthinspace\mkbibparens{#1}\addcolon}
\DeclareFieldFormat[article]{date}{#1}

% \AtEveryCitekey{\clearfield{month}}
% Replace number by month if number is undefined:
% ==>
\DeclareSourcemap{
    \maps[datatype=bibtex]{
        \map{
            \pertype{article}
			\step[notfield=number, final]
			\step[fieldsource=month, fieldtarget=number]
            \step[fieldset=month, null]
            \step[fieldset=month_numeric, null]
        }
    }
}
% <==

% Adjust formatting of back-references
% ==>
\DefineBibliographyStrings{english}{%
	backrefpage  = {},	% originally ``cit. on p.''
	backrefpages = {}	% originally ``cit. on pp.''
}
\DefineBibliographyStrings{ngerman}{%
	backrefpage  = {},	% originally ``Siehe Seite''
	backrefpages = {}	% originally ``Siehe Seiten''
}
\renewcommand*{\finentrypunct}{}
\renewbibmacro*{pageref}{%
	\addperiod
	\iflistundef{pageref}
	{}
	{\printtext[brackets]{%
			\ifnumgreater{\value{pageref}}{1}
				{\bibstring{backrefpages}}
				{\bibstring{backrefpage}}%
			\printlist[pageref][-\value{listtotal}]{pageref}%
		}%
	}%
}
% <==

% Move notes to the end of an entry by copying the ``note'' information into ``addendum'': -->
\DeclareSourcemap{
	\maps[datatype=bibtex]{
		% Copy values of the Mendeley-created ``annote'' field to the ``note'' field:
		\map[overwrite]{
		 	\step[fieldsource=annote]
		 	\step[fieldset=note, origfieldval, append]
			\step[fieldset=annote, null]
		}
		\map{
			\step[fieldsource=note, final]
			\step[fieldset=addendum, origfieldval, final]
			\step[fieldset=note, null]
		}
	}
}
% \DeclareFieldFormat{addendum}{(#1)} % Enclose addendum/note in parentheses.
% <--

\AtEveryBibitem{%
	\ifentrytype{newspaper}%
		{\clearfield{addendum}}% then
		{\clearfield{month}}% else
}

% Add ``working paper'' as the default value for type of ``techreports'' ==>
% see https://tex.stackexchange.com/questions/212362/how-to-use-declaresourcemap-to-add-default-value-to-a-field
\DeclareSourcemap{
	\maps[datatype=bibtex]{
		\map{% Will overwrite fields without the ``overwrite'' option
			\pertype{techreport}
			\step[fieldset=type, fieldvalue={Working paper}]
		}
	}
}
% <==

% Add “doctoral dissertation” as the default value for type of “phdthesis” ==>
% see https://tex.stackexchange.com/questions/212362/how-to-use-declaresourcemap-to-add-default-value-to-a-field
\DeclareSourcemap{
	\maps[datatype=bibtex]{
		\map{% Will overwrite fields without the ``overwrite'' option
			\pertype{phdthesis}
			\step[fieldset=type, fieldvalue={Doctoral dissertation}]
		}
	}
}
% <==

% Remove superfluous ``The'' from journal names ==>
\DeclareSourcemap{ 
	\maps[datatype=bibtex]{
		\map{
			\step[fieldsource=journal, match={\regexp{^The\s}}, replace={}]
		}
	}      
}
% <==

% Replace (Mendeley's) month strings (``jan'', ``feb'', etc.) by full names ==>
\DeclareSourcemap{ 
	\maps[datatype=bibtex]{
		\map{
			\step[fieldsource=number, match={\regexp{^jan$}}, replace=\regexp{\\mkbibmonth\{1\}}]
			\step[fieldsource=number, match={\regexp{^feb$}}, replace=\regexp{\\mkbibmonth\{2\}}]
			\step[fieldsource=number, match={\regexp{^mar$}}, replace=\regexp{\\mkbibmonth\{3\}}]
			\step[fieldsource=number, match={\regexp{^apr$}}, replace=\regexp{\\mkbibmonth\{4\}}]
			\step[fieldsource=number, match={\regexp{^may$}}, replace=\regexp{\\mkbibmonth\{5\}}]
			\step[fieldsource=number, match={\regexp{^jun$}}, replace=\regexp{\\mkbibmonth\{6\}}]
			\step[fieldsource=number, match={\regexp{^jul$}}, replace=\regexp{\\mkbibmonth\{7\}}]
			\step[fieldsource=number, match={\regexp{^aug$}}, replace=\regexp{\\mkbibmonth\{8\}}]
			\step[fieldsource=number, match={\regexp{^sep$}}, replace=\regexp{\\mkbibmonth\{9\}}]
			\step[fieldsource=number, match={\regexp{^oct$}}, replace=\regexp{\\mkbibmonth\{10\}}]
			\step[fieldsource=number, match={\regexp{^nov$}}, replace=\regexp{\\mkbibmonth\{11\}}]
			\step[fieldsource=number, match={\regexp{^dec$}}, replace=\regexp{\\mkbibmonth\{12\}}]
		}
	}      
}
% <==
