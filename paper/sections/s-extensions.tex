%!TEX root = ../main.tex
%-------------------------------------------------------------------------------
\section{Extensions}\label{Extensions}
%-------------------------------------------------------------------------------
We are actively pursuing several extensions to the standard analysis of EKW models. For example, we draw on the methodological literature on robust-decision making and uncertainty quantification to account for the uncertainties within and outside the model \citep{Hansen.2015}. We also work with the German Institute for Economic Research and Statistics Norway to improve the available data for the calibration of the models. Again, we have concluded our preparatory work and actively seek input from domain experts for further improvements and joint publication.
%-------------------------------------------------------------------------------
\subsection{Robust decision-making}
%-------------------------------------------------------------------------------
The uncertainties involved in human capital investments are ubiquitous \citep{Becker.1964}. Individuals usually make investments early in life when they are still uncertain about their abilities and tastes. Their returns also depend on demographic, economic, and technological trends that only start to unfold years from now. However, the treatment of uncertainty in EKW models of human capital investment is very narrow. A model provides individuals with a formalized view of their economic environment and implies unique probabilities for all possible future events. Individuals have no fear of model misspecification.\\

\noindent In \citet{Eisenhauer.2020}, we address this shortcoming by formulating, implementing, and exploring an EKW model of robust human capital investment where individuals face risk within a model and ambiguity about the model \citep{Arrow.1951}. Ambiguity arises as individuals do not know the true model and consider a whole set of models as reasonable descriptions of their economic environment. Individuals fear model misspecification and thus seek robust decisions, i.e., decisions that perform well over the whole range of models.\\

\noindent We incorporate methods from robust optimization \citep{Ben-Tal.2009, Rahimian.2019, Wiesemann.2014} and robust Markov decision processes \citep{Iyengar.2005, Nilim.2005} that allow to construct decision rules that explicitly take potential model misspecification into account.
%-------------------------------------------------------------------------------
\subsection{Uncertainty quantification}
%-------------------------------------------------------------------------------
There are numerous sources of uncertainty in the policy forecasts produced by a calibrated EKW model. The model is subject to misspecification, its numerical implementation introduces approximation error, the data is subject to measurement error, and the calibrated parameters remain partly uncertain. However, economists display incredible certitude as they disregard all uncertainty \citep{Manski.2013} in their forecasts.\\

\noindent In \citet{Eisenhauer.2020d}, we draw on a rich literature in other disciplines where a proper accounting of the uncertainty in forecasts from complex computational models is mandatory \citep{Saltelli.2004, Saltelli.2008, Smith.2014}. However, uncertainty quantification for EKW models poses several unique challenges. They usually have a large number of uncertain and correlated parameters, and the quantity of interest is time-consuming to compute and a complex function of the model parameters. Using machine learning methods, we set up an emulator that approximates the full model but is fast to evaluate. We revisit the analysis of \citet{Keane.1994, Keane.1997} to showcase our approach in a well-known and empirically-motivated setting and characterize the uncertainty in their key findings.\\

\noindent We construct our approximating emulator using recent advances in surrogate modeling \citep{Forrester.2008} and machine learning \citep{Hastie.2008, Murphy.2012}.
%-------------------------------------------------------------------------------
\subsection{Model validation}
%-------------------------------------------------------------------------------
The validation of computational models is a prerequisite for their use in other disciplines \citep{Adams.2012, Oberkampf.2010}. However, it is extremely rare in economics as drastic regime shifts are seldom available in observational data and costly to implement in large-scale experiments.\\

\noindent In \citet{Bhuller.2018}, we calibrate an EKW model on Norwegian population panel data with nearly career-long earnings histories. Due to the richness of the data, we can validate the model using a mandatory schooling reform. Our data includes substantial geographic variation in compulsory schooling across Norway between 1960 and 1975 as mandatory schooling increased from seven to nine years at different points in time across municipalities. We split our data into a calibration and validation sample. We only use pre-reform data in our calibration, forecast the effect of increasing mandatory schooling by two years, and compare our forecast with the post-reform outcome. Doing so allows us to assess our model's ability to extrapolate individual responses outside the support of our calibration data. We use the validated model to gain insights into the underlying economic mechanisms that generate the effects of the policy and forecast the effects of several policy alternatives.
%-------------------------------------------------------------------------------
\subsection{Nonstandard expectations}
%-------------------------------------------------------------------------------
When economists analyze individual decision-making through the lens of an EKW model, they impose rational expectations. The subjective beliefs about the future correspond to the objective transition probabilities induced by the model.\\

\noindent In \citet{Eisenhauer.2020a}, we relax this assumption. We analyze and quantify the effect of biased expectations about wage growth in part-time employment on life cycle wage profiles of female workers. We design specific survey questions for the German Socio-Economic Panel \citep{Goebel.2019} and elicit the expected wage trajectories for full-time and part-time employment directly. Thus, we can incorporate the belief elicitation directly in our life cycle model.
