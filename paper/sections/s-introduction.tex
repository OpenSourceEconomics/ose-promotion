%!TEX root = ../main.tex
%-------------------------------------------------------------------------------
\section{Introduction}
%-------------------------------------------------------------------------------


Computational economic models specify an individuals objective and the institutional and informational constraints of their economic environment. They are used to quantify the importance of competing economic mechanisms in determining economic outcomes and forecast the effects of alternative policies before their implementation. We provide a platform for economists, mathematicians, and computational scientists to facilitate the transdisciplinary collaboration in the development, analysis, and application of computational economic models. Together, we expand the set of possible economic questions that we can address and improve the quality of our answers.

We provide numerous software libraries that we use in our research. We also host several events to facilitate the exchange of ideas with our collaborators; for example, we organize a monthly OSE Meetup and an annual OSE Retreat.

\paragraph{Transdisciplinary research requires common ground.} We organize our collaborations around an ensemble of versatile research codes that allow for the analysis of different classes of economic models. These codes are systematically developed, thoroughly tested, well documented, and actively used. Their architecture allows focusing on selected mathematical and computational challenges within a deliberately simplified model while offering the immediate ability to scale up to a full-fledged research project. Please visit our website (https://open-econ.org) or contact me directly for additional information.

\paragraph{Transdisciplinary research requires training and preparation.} We create a new Master and Ph.D. course in the upcoming winter term for scientific computing. The course is tailored to economists but open to junior researchers across the whole transdisciplinary research area. The goal is to establish a common methodological toolkit and discuss its use-cases in computational economic models. To that end, we initially review standard numerical challenges such as optimization, function approximation and integration, and uncertainty quantification. We move on to presentations on our groups research codes that enable the flexible specification of micro- and macroeconomic models. We conclude by presenting the core packages of our analysis pipeline for (robust) optimization and uncertainty quantification. All packages are open source and under constant development, and the presentations have the explicit goal to recruit future collaborators and contributors. We invite the Life \& Medical Sciences Institute (LIMES) and the Institute for Numerical Simulation (INS) for guest lectures. In the near future, we hope to administer this course in equal partnership with the LIMES and INS in support of the transdisciplinary research profile of the University of Bonn as a whole.

\paragraph{Transdisciplinary research demands opportunities for personal interaction and public outreach.} We host an international three-day conference (including a one-day networking event) centered around computational methods and models relevant for economists. We recruit senior and junior researchers from our transdisciplinary research area as participants and presenters. Thus, we will set up presentations at the beginning of each day that rovide the background motivation for the application of the models within economics. But then focus on the challenges arising from their mathematical formulation and computational implementation for the rest of the day. We showcase our efforts to quantify the inherent uncertainties in the analysis of computational economic models and to explore ways to communicate them to the broader public.

\noindent At this point, we are about 35 individuals with different level of involvement from all levels of the academic hierarchy ranging from professors to bachelor students and we just received funding by the Excellence Strategy to scale up our initiatives.\\
