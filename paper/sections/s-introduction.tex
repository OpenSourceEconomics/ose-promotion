%!TEX root = ../main.tex
%-------------------------------------------------------------------------------
\section{Introduction}
%-------------------------------------------------------------------------------
I created a platform for economists, mathematicians, and computational scientists to facilitate the transdisciplinary collaboration in the development, analysis, and application of computational economic models. Our goal is to expand the set of possible economic questions that we can address and improve the quality of our answers.\\

\noindent I organize this collaboration around an ensemble of versatile research codes that allow for the analysis of different classes of economic model. These codes are systematically developed, thoroughly tested, well documented, and actively used. Their architecture allows focusing on selected mathematical and computational challenges within a deliberately simplified model while offering the immediate ability to scale up to a full-fledged research project.\\

\noindent At this point, we are about 35 individuals with different level of involvement from all levels of the academic hierarchy ranging from professors to bachelor students and we just received funding by the Excellence Strategy to scale up our initiatives.\\

\noindent So, let me now tell you what we do, how we do it, and with whom,

\subsubsection*{Research codes}

It is useful to distinguish two groups of research codes. The first group implements particular economic models, while the second group provides model-agnostic capabilities for their analysis.

\begin{itemize}
\item \textbf{respy} finite discrete-Markov decision process, often used to model human capital investment decisions
\item \textbf{pydsge},
\item \textbf{ruspy} infinite horizon Markov decision problem, often used to model firm investment behaviors
\end{itemize}

We created a pipeline for the analysis of these models.

\begin{itemize}
\item \textbf{estimagic}, global optimization for the calibration of computational economic models
\item \textbf{econsa}, uncertainty analysis
\item \textbf{robupy} robust optimization
\end{itemize}

\noindent My own involvement in the development and application of these research codes varies considerably. While I am an active developer for some, I am guiding development by my students in other, and for some I do not have much of a say at all. I am now more and more moving into the supervisory activity.\\

\noindent While I wrote the first release for several of these packages, I am nowadays mostly guiding their development thorough the identification of research needs. This is only possible while maintaining the quality of the software architecture and the computational implementation by adopting a sound software engineering workflow and training capable contributors.

\subsubsection*{Workflow}


If you are interested, we have our website at open-econ.org


%-------------------------------------------------------------------------------
\subsubsection*{Collaboration}
%-------------------------------------------------------------------------------
We have now used this platform to start several collaborations.
