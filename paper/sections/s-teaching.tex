%!TEX root = ../main.tex
%-------------------------------------------------------------------------------
\section{Teaching}
%-------------------------------------------------------------------------------

How does this look like in practice?

\paragraph{Scientific computing course} standard stuff, respy for labor, and then project-based learning as contributors, guest lectures by the Institute of numerical simulation and the LIMES institute. So the course is part of the new transdiciplinary research initiative at the University of Bonn and we will also have guest lectures by the INS and LIMES. The goal is to establish a common technological toolkit and discuss is use-cases in computational economic models. We first provide an overview of basic numerical methods for optimization, numerical integration, approximation methods, and uncertainty quantification. We then deepen our understanding of each of these topics in the context of a dynamic model of human capital accumulation using \textbf{respy}. We conclude by showcasing basic software engineering practices such as the design of a collaborative and reproducible development workflow, automated testing, and high-performance computing.\\

\noindent Recruiting contributors for our software packages is an explicit goal and so students will work on projects that are tightly related to our software packages

\paragraph{Data science}

\paragraph{Nuvolos} I build both courses on the Novulus-cloud as an integrated teaching and research platform. It provides a simple, browser-based environment in the cloud that gives me complete control over students computational environment and simplifies the dissemination of teaching material. This allows for a project-based learning where students can seamlessly work in groups and have the computation resourcse available to go from a simple student project to a full-fledged research application. This infrastructure scales well with the number of participants.

\paragraph{Concluding remarks}

\begin{itemize}
\item experience in teaching labor economics and human capital formation
\item adjust content of course
\end{itemize}
