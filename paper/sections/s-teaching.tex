%!TEX root = ../main.tex
%-------------------------------------------------------------------------------
\section{Teaching}
%-------------------------------------------------------------------------------

\subsubsection*{Training} Team meeting, Summer School, Retreat, conferences, courses, hackathons, meetup


How does this look like in practice?

\paragraph{Scientific computing} standard stuff, respy for labor, and then project-based learning as contributors, guest lectures by the Institute of numerical simulation and the LIMES institute. So the course is part of the new transdiciplinary research initiative at the University of Bonn and we will also have guest lectures by the INS and LIMES. The goal is to establish a common technological toolkit and discuss is use-cases in computational economic models. We first provide an overview of basic numerical methods for optimization, numerical integration, approximation methods, and uncertainty quantification. We then deepen our understanding of each of these topics in the context of a dynamic model of human capital accumulation using \textbf{respy}. We conclude by showcasing basic software engineering practices such as the design of a collaborative and reproducible development workflow, automated testing, and high-performance computing.\\

\noindent Recruiting contributors for our software packages is an explicit goal and so students will work on projects that are tightly related to our software packages

\paragraph{Data science}

\paragraph{Nuvolos} We build the course on the Nuvolos.cloud (https://nuvolos.cloud) as an integrated research and teaching platform. The platform provides a simple, browser-based environment in the cloud that gives instructors complete control over students’ computational environment and simplifies the dissemination of teaching material. This service is particularly useful during remote teaching as the debugging of any hardware or software issues faced by individual students is particularly challenging. High-performance computing resources in support of our ongoing transdisciplinary research projects are available.

\paragraph{Concluding remarks}

\begin{itemize}
\item experience in teaching labor economics and human capital formation
\item adjust content of course
\end{itemize}
