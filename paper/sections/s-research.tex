%!TEX root = ../main.tex
%-------------------------------------------------------------------------------
\section{Research}\label{Transdisciplinary research}
%-------------------------------------------------------------------------------




\subsection{Software}

It is useful to distinguish two groups of research codes. The first group implements particular economic models, while the second group provides model-agnostic capabilities for their analysis.

\begin{itemize}
\item \textbf{respy} finite discrete-Markov decision process, often used to model human capital investment decisions
\item \textbf{pydsge},
\item \textbf{ruspy} infinite horizon Markov decision problem, often used to model firm investment behaviors
\end{itemize}

We created a pipeline for the analysis of these models.

\begin{itemize}
\item \textbf{estimagic}, global optimization for the calibration of computational economic models
\item \textbf{econsa}, uncertainty analysis
\item \textbf{robupy} robust optimization
\end{itemize}

\subsubsection*{Workflow}


If you are interested, we have our website at open-econ.org, GitHub Organization, Gitflow Workflow, Code reviews, testing harness, continuosu integratoin


\paragraph{respy} The respy (respy, 2018) package is our most advanced research code. It is capable of solving, simulating, and estimating a whole class of economic models commonly used in labor economics. The analysis of such models allows to study the driving forces behind observed inequalities in a variety of economic outcomes. They are used to assess the relative importance of competing economic mechanisms and to predict the effects of public policies. Their mathematical formulation corresponds to a finite-horizon Markov decision process whose computational implementation poses several numerical challenges.These include integration, function approximation, global optimization, and uncertainty quantification.

\noindent Going forward, we will treat \textbf{respy} as a research project in its own right. The code base is now so mature that we need expert support to further improve its numerical components and unlock the potential for large scale parallelism.






\subsection{Projects}
