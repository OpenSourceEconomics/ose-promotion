%!TEX root = ../main.tex
%-------------------------------------------------------------------------------
\section{Research}\label{Research}
%-------------------------------------------------------------------------------
Human capital is the knowledge, skills, competencies, and attributes embodied in individuals that facilitate the creation of personal, social, and economic well-being \citep{OECD.2001}. Differences in human capital attainment are a major determinant of inequality in a variety of life outcomes such as labor market success and educational attainment across and within countries. However, our understanding about the driving forces behind these differences is still limited.

\noindent In my research, I shed new light on the underlying economic mechanisms by combining innovative economic modeling with access to unique data sets and advanced computational methods.\\

\noindent I organize my research agenda around the class of Eckstein-Keane-Wolpin (EKW) models \citep{Adda.2017, Blundell.2016, Keane.1997}. These models specify the objective of individuals, their economic environment, and the institutional and informational constraints under which they operate. Calibration of the model to observed data on individual decisions and experiences allows quantifying the importance of competing economic mechanisms in determining economic outcomes and forecasting the effects of policy proposals \citep{Wolpin.2013}.\\

\noindent My flagship research code \textbf{respy} allows for the flexible specification, simulation, and estimation of this class of economic models. The mathematical formulation corresponds to a finite-horizon Markov decision process. My research agenda emphasizes a the class models and does not emphasize its underlying economic questions. This is due to the transdiciplanrity of my research where the math and its computational implementation provides the common ground to work together. My example project provided one application of this model. We just completed the second release of our software and now have a stable set of baseline capabilities that we now exploit in a first tranche of papers.\\

\noindent Going forward, we will treat \textbf{respy} as a research project in its own right. The code base is now so mature that we need expert support to further improve its numerical components and unlock the potential for large scale parallelism.

\paragraph{Projects} With the help of \textbf{respy} we are able to solve the whole class of economic models and thus here is a list of my research projects all based on it.

\paragraph{Concluding remarks} Not all just EKW models, but the focus. However, validation is important during my work there, so that is why I am very interested in it on the context of structural behavioral economics. Among them a project with Armin Falk. We are also now cooperating with epidemiologist for an integrated assessment model on pandemic and economics.\\

\noindent Most importantly, there is a time to plant and a time to reap. But I wanted this to exist to badly, that I lost sight of the other pressures in our profession. But now this is exists and is so unique, that I want to prove its success and contribute to the TUM ecosystem.
